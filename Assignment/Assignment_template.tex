\documentclass[12pt]{amsart}

\usepackage{amsfonts,latexsym,amsthm,amssymb,amsmath,amscd,euscript}

\usepackage{ragged2e} %To align text.
\usepackage{framed} %To add frame
\usepackage{setspace} %Setting the line space 
\usepackage{times} %Choose Font
\usepackage{fullpage} %Automatically set the margin

\usepackage{hyperref}
    \hypersetup{colorlinks=true,citecolor=blue,urlcolor =black,linkbordercolor={1 0 0}}

\newenvironment{statement}[1]{\smallskip\noindent\color[rgb]{1.00,0.00,0.50} {\bf #1.}}{}
\allowdisplaybreaks[1]

%Below are the theorem, definition, example, lemma, etc. body types.

\newtheorem{theorem}{Theorem}
\newtheorem*{proposition}{Proposition}
\newtheorem{lemma}[theorem]{Lemma}
\newtheorem{corollary}[theorem]{Corollary}
\newtheorem{conjecture}[theorem]{Conjecture}
\newtheorem{postulate}[theorem]{Postulate}
\theoremstyle{definition}
\newtheorem{defn}[theorem]{Definition}
\newtheorem{example}[theorem]{Example}

\theoremstyle{remark}
\newtheorem*{remark}{Remark}
\newtheorem*{notation}{Notation}
\newtheorem*{note}{Note}

% You can define new commands to make your life easier.
\newcommand{\BR}{\mathbb R}
\newcommand{\BC}{\mathbb C}
\newcommand{\BF}{\mathbb F}
\newcommand{\BQ}{\mathbb Q}
\newcommand{\BZ}{\mathbb Z}
\newcommand{\BN}{\mathbb N}

% We can even define a new command for \newcommand!
\newcommand{\nc}{\newcommand}

% If you want a new function, use operatorname to define that function (don't use \text)
\nc{\on}{\operatorname}
\nc{\Spec}{\on{Spec}}

\title{Problem Set Template $2$} 

\date{\today}
\begin{document}

\maketitle

\vspace*{-0.25in}
\centerline{Suryansh Shukla}
\centerline{home address 1}
\centerline{home address line 2}
\centerline{\href{myemail@gmail.com}{{\tt myemail@gmail.com}}}
\vspace*{0.15in}

%%Spacing %%
\begin{doublespace} %To add the double space 
\end{doublespace}
\singlespace %single space whole document 
\doublespace %Double Space whole document
%%%%
\textbf{Bullets}
\begin{verbatim} 
    \begin{enumerate}
        \item Do Math 55 problemset.
        \item Do Math 55 problemset.
        \item Do Math 55 problemset.
    \end{enumerate}
\end{verbatim}
\textbf{Output}
\begin{enumerate}
    \item Do Math 55 problemset.
    \item Do Math 55 problemset.
    \item Do Math 55 problemset.
\end{enumerate}
\textbf{Defination }
\begin{verbatim}
    \begin{defn}
        Defination Body.
        \end{defn}
\end{verbatim}
\textbf{Output}  
\begin{defn}
    Defination body.
\end{defn}

\textbf{Proof}
\begin{verbatim}
    \begin{proof}
        Add proof body here. 
    \end{proof}
\end{verbatim}
\textbf{Output}
\begin{proof}
    Add proof body here. 
\end{proof}

\textbf{Frame \& problem set}
\begin{verbatim}
    \begin{framed}
        \begin{statement}{Problem 1}
            Add the problem statement here.
        \end{statement}
    \end{framed}
\end{verbatim}

\textbf{Output}
\begin{framed}
    \begin{statement}{Problem 1}
        Add the problem statement here.
    \end{statement}
\end{framed}

\textbf{Equations}
\\
ADD * to not include equation numbers
\begin{verbatim}
\begin{align} %To add several equations
    0 &= 0 + 0 + 0 + 0 + \dots\\
    &= (1 - 1) + (1 - 1) + \dots \\
    &= 1 + (-1 + 1) + (-1 + 1) + \dots \\
    &= 1 + 0 + 0 + 0 \dots \\
    &= 1.
\end{align}
\end{verbatim}
\textbf{Output}
\begin{align} %To add several equations
    0 &= 0 + 0 + 0 + 0 + \dots\\
    &= (1 - 1) + (1 - 1) + \dots \\
    &= 1 + (-1 + 1) + (-1 + 1) + \dots \\
    &= 1 + 0 + 0 + 0 \dots \\
    &= 1.
\end{align}
\textbf{Alternative Command for adding single equation}
\begin{verbatim}
    \begin{equation}
        E = m*c^2
    \end{equation}
\end{verbatim}
\textbf{Output}
\begin{equation}
    E = m*c^2
\end{equation}

\textbf{Text Alignment}
\begin{verbatim}
    \usepackage{ragged2e}
    \justify
    \Centering
    \RaggedLeft
    \RaggedRight
\end{verbatim}

\textbf{Table}
\begin{verbatim}
    \begin{center}
        \begin{tabular}{ |p{3cm}||p{3cm}|p{3cm}|p{3cm}| }
            \hline \hline
         cell1 & cell2 & cell3 \\ 
         \hline\hline
         cell4 & cell5 & cell6 \\ 
         \hline 
         cell7 & cell8 & cell9\\
         \hline  
        \end{tabular}
        \end{center}
\end{verbatim}
\textbf{Output}
\begin{center}
    \begin{tabular}{ |p{3cm}||p{3cm}|p{3cm}|p{3cm}| }
        \hline \hline
     cell1 & cell2 & cell3 \\ 
     \hline\hline
     cell4 & cell5 & cell6 \\ 
     \hline 
     cell7 & cell8 & cell9\\
     \hline  
    \end{tabular}
    \end{center}
\end{document}

