\documentclass[12pt]{amsart}

%Below are some necessary packages for your course.
\usepackage{amsfonts,latexsym,amsthm,amssymb,amsmath,amscd,euscript}
\usepackage{framed}
\usepackage{fullpage}
\usepackage{hyperref}
    \hypersetup{colorlinks=true,citecolor=blue,urlcolor =black,linkbordercolor={1 0 0}}

\newenvironment{statement}[1]{\smallskip\noindent\color[rgb]{1.00,0.00,0.50} {\bf #1.}}{}
\allowdisplaybreaks[1]

%Below are the theorem, definition, example, lemma, etc. body types.

\newtheorem{theorem}{Theorem}
\newtheorem*{proposition}{Proposition}
\newtheorem{lemma}[theorem]{Lemma}
\newtheorem{corollary}[theorem]{Corollary}
\newtheorem{conjecture}[theorem]{Conjecture}
\newtheorem{postulate}[theorem]{Postulate}
\theoremstyle{definition}
\newtheorem{defn}[theorem]{Definition}
\newtheorem{example}[theorem]{Example}

\theoremstyle{remark}
\newtheorem*{remark}{Remark}
\newtheorem*{notation}{Notation}
\newtheorem*{note}{Note}

% You can define new commands to make your life easier.
\newcommand{\BR}{\mathbb R}
\newcommand{\BC}{\mathbb C}
\newcommand{\BF}{\mathbb F}
\newcommand{\BQ}{\mathbb Q}
\newcommand{\BZ}{\mathbb Z}
\newcommand{\BN}{\mathbb N}

% We can even define a new command for \newcommand!
\newcommand{\nc}{\newcommand}

% If you want a new function, use operatorname to define that function (don't use \text)
\nc{\on}{\operatorname}
\nc{\Spec}{\on{Spec}}

\title{Math 55a, Problem Set $n$} % IMPORTANT: Change the problemset number as needed.
\date{\today}

\begin{document}

\maketitle

\vspace*{-0.25in}
\centerline{Suryansh Shukla}
% Just so that your CA's can come knocking on your door when you don't hand in that problemset on time...
\centerline{home address 1}
\centerline{home address line 2}
\centerline{\href{myemail@gmail.com}{{\tt myemail@gmail.com}}}
\vspace*{0.15in}

\begin{framed}
  Note: Add problem set here to be in frame. 
\end{framed}

\begin{statement}{1}
    Write here proble set 1
\end{statement}

\begin{proof}
    Write the solution needed. \\
    Use definations \\
    Use lemmas \\
    Use proofs 
    \begin{defn}
    write definations here.

    \end{defn}

    Write equations.

    \begin{align*}
        0 &= 0 + 0 + 0 + 0 + \dots\\
        &= (1 - 1) + (1 - 1) + \dots \\
        &= 1 + (-1 + 1) + (-1 + 1) + \dots \\
        &= 1 + 0 + 0 + 0 \dots \\
        &= 1.
    \end{align*}

    If you need to list things, use {\tt enumerate} or {\tt itemize}; e.g. Daily Schedule:
    \begin{enumerate}
        \item Do Math 55 problemset.
        \item Do Math 55 problemset.
        \item Do Math 55 problemset.
    \end{enumerate}
    And {\tt itemize} gives you bullet points.
\end{proof}

\begin{statement}{2}
 Problem set 2
\end{statement}
\begin{proof}
   Write proof.
\end{proof}

\begin{proof}
    abc
\end{proof}
\end{document}
