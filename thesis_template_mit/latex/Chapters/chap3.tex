\chapter{RESEARCH METHODOLOGY}\thispagestyle{EmptyHeader}
\label{chp:3}
Description of methods, approaches, case and materials as relevant that would be used in the research study. The method and approach identified should follow logically and must be appropriate for the Research Question(s) that has been identified and formulated. Instruments or approaches used for data collection, analysis, and results interpretation. Description of how and where the study will be conducted.  Sampling if relevant including sampling frame, and size. Credibility or authenticity of the data collected. Brief description of ethical issues related to the study and the ways in which it would be managed. Description of health and safety issues (if any) and the ways in which it should/would be managed. List of summary of IP, patents, copyrights if any.
\section{Figures}

Figures, like tables are printed within the body of the text at the center of the frame and labelled according to the chapter in which they appear. Thus, for example, figures in Chapter 3 are numbered sequentially: Figure 3.1, Figure 3.2, you can refer to a diagram via label like \ref{diag:sample}.
Figures, unlike text or tables, contain graphs, illustrations or photographs and their labels are placed at the bottom of the figure rather than at the top (using the same format used for tables). If the figure occupies more than one page, the continued figure on the following page should indicate that it is a continuation: for example: ‘Figure 3.7, continued’. If the figure contains a citation, the source of the reference should be placed at the bottom, after the label.
To insert label below a figure, click “Insert Caption” under the “References” tab and select “Figure” in the dropdown list. Click “Update Table” to update the List of Figures.

\begin{figure}[ht]
	\centering
	\includegraphics{images/blockdiagram.jpg}
	\caption{Sample Diagram 1}
	\label{diag:sample}
\end{figure}



\clearpage
\newpage


\begin{figure}[b]
	\centering
	\includegraphics[width=\linewidth]{images/phd.png}
	\caption{Sample Diagram 2}
	\label{diag:sample2}
\end{figure}

\clearpage
\newpage


\begin{sidewaysfigure}
	\centering
	\includegraphics[width=\linewidth]{images/phd.png}
	\caption{Sample Diagram 2 sideways}
	\label{diag:sample3}
\end{sidewaysfigure}


