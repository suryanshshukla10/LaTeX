
%% This is an example first chapter.  You should put chapter/appendix that you
%% write into a separate file, and add a line \include{yourfilename} to
%% main.tex, where `yourfilename.tex' is the name of the chapter/appendix file.
%% You can process specific files by typing their names in at the
%% \files=
%% prompt when you run the file main.tex through LaTeX.
\chapter{INTRODUCTION}\thispagestyle{EmptyHeader}
\label{chp:1}

The body of the text should be typed with double spacing. Single-spacing is only permitted in tables, long quotations, footnotes, citation and in the bibliography.
Beginning of the first line of each paragraph should have 0.5cm indentation.

\section{Background}
Typically describes the background to the topic of study, reason why this topic was chosen, and why it is relevant to be researched. 
First topic in each chapter should numbered with “chapter number”.1. Use Heading 2 or h2 for title and for table of content TOC3 must be used.
\section{Motivation}
Motivation for the study.
\section{Problem Statement}
Brief description of the problem under study. 
\section{Research Questions}
Research questions are formed or synthesised through the literature review and clearly stated. Should have the potential to contribute to knowledge in the field of study.

\begin{enumerate}
	\item Question 1?
	\item Question 2?
	\item Question 3?
	\item Question 4?
\end{enumerate}
\section{Hypothesis}
Hypothesis, if you have any. Otherwise, you can ignore this section.
It is not a good idea to go beyond third subtitle i.e. three levels of subtitle.

\begin{enumerate}
	\item Hypothesis 1?
	\item Hypothesis 2?
	\item Hypothesis 3?
	\item Hypothesis 4?
\end{enumerate}
\section{Research Objectives}
Research Objectives are formed or synthesized through the literature review and clearly stated. Should have the potential to contribute to knowledge in the field of study
\begin{enumerate}
	\item Objective 1?
	\item Objective 2?
	\item Objective 3?
	\item Objective 4?
\end{enumerate}

\section{Scope and limitations of the study}
States what is included and what is not included in the thesis with validation.
\section{Significance of the study}
Explains why this study is important, and worth researching. Includes the benefits, expected outcomes, and applications
\clearpage
\newpage